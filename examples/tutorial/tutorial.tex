\documentclass{article}
\usepackage[a4paper, total={7in, 9in}]{geometry}
\usepackage[utf8]{inputenc}
\usepackage{graphicx}
\usepackage{tikz}
\usepackage{listings}
\usepackage{xcolor}
\usepackage[hidelinks]{hyperref}
\hypersetup{
    colorlinks,
    linkcolor={black!70},
    citecolor={black!70},
    urlcolor={blue!60!black}
}
\usetikzlibrary{
  calc,arrows,automata,backgrounds,decorations,decorations.pathreplacing,decorations.markings,decorations.pathmorphing,fit,matrix,decorations.text,
  positioning,shapes,shapes.geometric,shapes.symbols}%
\usepackage{pgfuml} 

\begin{document}
\title{Pymoult Tutorial}
\date{}
\maketitle

\section{Prerequisites}

This tutorial requires a working installation of Pymoult. You can
follow the instructions
\href{https://bitbucket.org/smartinezgd/pymoult/wiki/Home}{here} to
install it.

Or you can download a \href{https://bitbucket.org/smartinezgd/pymoult/wiki/Virtual\%20Machine}{Virtual Machine} with Pymoult pre-installed and
ready to use. If chose this option, make sure Pymoult is up to date
following the instructions on the VM's page.  

\section{The Tutorial}

For each exercise, you can find the solution in a \texttt{solution.py}
file. It provides an answer to the exercise. Keep in mind that there
can be more than one way to apply the updates.  To execute the
solution, run \texttt{./test.sh solution} in the proper directory.
The documentation of Pymoult can be found
\href{https://bitbucket.org/smartinezgd/pymoult/wiki/Pymoult\%20manual}{here}
and examples of use of mechanisms can be found in the directory
\texttt{pymoult/examples/tests}.

\section{Step 1: Using the High-level API}

The first step of our tutorial will show how to use the predefined
updates classes provided by the high-level API of Pymoult. The
following exercises can be found in the \texttt{step1} folder. 

\subsection{Exercise 1: Hello World}

Our first update will update a small \textit{hello world}
program. Look at the application code: a loop calls the
\texttt{hello} function every 2 seconds.

Now, open the \texttt{update.py} file. See how it begins with
\texttt{\#parsed}. This keyword tells Pymoult that this is an update
file. Notice also how a manager is started during the update.  

Complete the \texttt{update.py} file by defining a
\texttt{hello\_v2} function that prints \textit{hello <your
  name>}. Then create an update object using the
\texttt{SafeRedefineUpdate} class from
\texttt{pymoult.highlevel.updates} and send to the manager by adding
\texttt{manager.add\_update(<name of your update object>)}. 

When you think your update is ready, run \texttt{./test.sh} and look
at the result.


\subsection{Exercise 2: Tic toc, goes the clock}

In this exercise, use what you learned in the previous exercise to
update the functions \texttt{tic} and \texttt{tac} of the
application.

Open the \texttt{application.py}, this application calls
\texttt{tic} then \texttt{tac} each second. Note that the
application starts a manager before calling its main function. You can
use that manager when updating by calling \\
\texttt{main.manager.add\_update(<update>)} where \texttt{main}
is the module captured by \texttt{sys.modules["\_\_main\_\_"]}.

Now, create an \texttt{update.py} file (you can copy the update file
from previous example) and create two functions \texttt{tic\_v2} and
\texttt{tac\_v2}. Then using two instances of
\texttt{SafeRedefineUpdate}, update \texttt{tic} and
\texttt{tac} to \texttt{tic\_v2} and \texttt{tac\_v2}.

Run \texttt{./test.sh} when your update is ready

\subsection{Exercice 3: Updating Data}

In this exercise, you will use an instance of
\texttt{EagerConversionUpdate} to update the class \texttt{Data} of
the application.

Create an \texttt{update.py} file and write a new version of the
\texttt{Data} class where it has an \textit{origin} attribute. When an
instance of \texttt{Data\_v2} is printed, it should display
\textit{<value> from <origin>}.

Then, write a \texttt{tranformer(data)} function that takes an
instance of \texttt{Data} (old version) and adds a \textit{origin}
attribute to it so that it becomes compatible with
\texttt{Data\_v2}. After that, you can create an instance of
\texttt{EagerConversionUpdate} and send it to the manager to update
\texttt{Data}.

\section{Step 2: Combining update classes} 

The second step of our tutorial shows how to combine predefined update
classes to design more complex updates. The following exercises can be
found in the \texttt{step2} folder.

\subsection{Exercise 1: Back to updating Data}

This exercise uses the same \texttt{Data} class as last exercise. The
application maintains a list of \texttt{Data} following the principle
of the Fibonacci sequence: the new element of the list is the sum of
the two last elements of the list.

As in the previous exercise, origin is added to the \texttt{Data}
class, which requires modifying the \texttt{sum} function. Open
\texttt{update.py} and use the update classes of the High-level API of
Pymoult to update \texttt{Data}  and \texttt{sum}.

Notice that in \texttt{Data\_v2}, the \textit{origin} argument of the
constructor is optional. This is because \texttt{Data} and
\texttt{sum} are updated separately: the old version of \texttt{sum}
may create an instance of \texttt{Data\_v2} without supplying the
\textit{origin} argument.

\subsection{Exercise 2: Series of integer}

This application generates ten thousand integers then creates a serie
of 5 randomly picked integers every two seconds.

Write an update that applies two modifications to the application:

\begin{enumerate}

\item Update the \texttt{Integer} class so that all integers are multiplied by 2.

\item Redefine the \texttt{\_\_str\_\_} method of \texttt{Serie} so that
  it shuffles the integers before returning its content.

\end{enumerate}

Since there are a lot of integers in the application, use the Lazy
strategy for updating them.

Hint: you can shuffle a list by calling \texttt{random.shuffle(<list>)}.

\section{Step 3: Developing custom update classes}

The third step of our tutorial shows how to develop custom update
classes by combining mechanisms from the Low-level API of Pymoult. The
following exercises can be found in the \texttt{step3} folder.

\subsection{Exercice 1: Hello again}

Here, you will solve the first exercise of the first step again, but
by defining your own update Class. 

Open the \texttt{update.py} file. You need to complete the code of the
\texttt{CustomUpdate} by filling up its methods:

\begin{itemize}
\item \texttt{wait\_alterability} must return \textit{True} when
  \texttt{hello} is quiescent

\item \texttt{resume\_hook} is called when resuming the program

\item \texttt{apply} must redefine \texttt{hello} to its new version

\end{itemize}

Then, create an instance of \texttt{CustomUpdate} and send it manager.


\subsection{Exercice 2: The return of the Data update}

Remember the first exercise of step 2? The new version of the
\texttt{Data} had to define a default value for the \textit{origin}
attribute because the two parts of the update where independent. In
this exercise, you will write a custom update that will redefine
\texttt{sum} and update \texttt{Data} in a single go.

To update \texttt{Data}, you can use the following mechanisms from the
low-level API:

\begin{itemize}
\item \texttt{DataAccessor(class,"immediate")}, creates an iterator
  over all instances of \textit{class} that exist when the data
  accessor is created.

\item \texttt{redefineClass(module,class,class_v2)}, redefines
  \textit{class} from \textit{module} to \textit{class\_v2}.

\item \texttt{updateToClass(obj,class,class\_v2,transformer)}, updates
  \textit{obj} of class \textit{class} to \textit{class\_v2} and
  applies \textit{transformer}.

\end{itemize}

\subsection{Exercice 3: The clock never stops ticking}

In the second exercise of the first step, you used safe redefinition
to update functions \texttt{tic} and \texttt{tac}. While this works,
it does not guarantee that the update won't happen between calls to
\texttt{tic} and \texttt{tac}.

Open the \texttt{application.py} file and notice the static update
point placed at beginning of \texttt{main}'s loop. You will use it as
alterability criteria to ensure that the following trace is not possible:

\textit{tic,tac,tic,tac2,tic2,...}


Notice that the \texttt{main} function is called in a
\texttt{DSU\_Thread}. This Pymoult specific Thread is an extension of
the \texttt{threading.Thread} class of Python. It is required for
using static points.


To use the static point as alterability criteria, you will need to
call \texttt{setupWaitStaticPoints} during the first setup step of the
update (method \texttt{preupdate\_setup} and
gg\texttt{cleanFailedStaticPoints} in the
\texttt{clean\_failed\_alterability}.

\subsection{Exercice 4: Rebooting threads}

Let's take the \textit{Hello world} example and use a different method
to update it. In this exercise, you will use Thread rebooting.

Place a static update point in the code of the application and write
an update that reboots \texttt{thread} to update the \texttt{hello}
function.
 
\begin{itemize}
\item \texttt{switchMain(thread,new\_main)} changes the main function of DSU\_Thread thread to new\_main.
\item \texttt{resetThread(thread)} reboots thread (it must be a DSU\_Thread)  

\end{itemize}


\section{Step 4: Complex updates}

In the last step of our tutorial, you will use everything you learned
in the previous steps to develop complex updates. The
following exercises can be found in the \texttt{step4} folder.

\subsection{Exercice 1: Simultaneous update}

In this application, a shared variable is read and written by two
functions (\texttt{write\_shared} and \texttt{read\_shared}). We want
to update the application so that the variable stores the time when it
was written. This requires the two functions to be updates at the same
time. Indeed, if the \texttt{write\_shared} write the variable in the
new format, the old version of \texttt{read\_shared} would crash when
reading it.

In this example, we will use the same strategy as in DynamOS (Dynamic
and Adaptive Updates of Non-Quiescent Subsystems in Commodity
Operating System Kernels, Eurosys 2007): In a first phase, we well
redefine \texttt{write\_shared} and \texttt{read\_shared} to
intermediate functions that can call both old and new versions:

\begin{verbatim}
def interm_write():
    if update:
        <code of write_shared_v2>
    else:
        <code of write_shared>
\end{verbatim}
 
When both \texttt{write\_shared} and \texttt{read\_shared} are
replaced by their intermediate version, the global variable
\texttt{update} is set to true. The application will be behave as if
it where in its new version. Both functions can, then, be redefined to
\texttt{write\_shared\_v2} and \texttt{read\_shared\_v2}.


\subsection{Exercice 2: Web Server}

The application here is a web server serving small predefined pages on
localhost:8080. After running \texttt{./test.sh}, you can go to
\url{http://localhost:8080/hello}, \url{http://localhost:8080/banana},
\url{http://localhost:8080/coconut}. For this example, the update will
not be summoned until you press a key.

The application uses the \texttt{server} python library (embedded by
the Python interpreter). Before updating, familiarize yourself with
the code of the application.

The update will add authentication to the server. It will require
users to log in and customize the messages in the served pages to the
user currently logged in. All the necessary code for this can be found
in the \texttt{update.py} file.

To apply this update, we advise that you proceed in two steps. As a
first step, update the \texttt{Page} and the \texttt{Session}
classes. Then, update the \texttt{Webserver} class and redefine the
\texttt{do\_GET} method of the Handler class. You will need to generate
the new version with the \texttt{create\_new\_do\_GET} function in
\texttt{update.py}.

Hint: You may use a static update point to indicate the alterability
of the web server.

\end{document}
